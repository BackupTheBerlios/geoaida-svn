

\section{BottomUp}
\subsubsection{Aufgabe des BottomUp-Operators}
Der BottomUp-Operator hat die Aufgabe eine ihm �bergebene Knotenliste
zu guppieren. Jede Gruppe enth�lt dabei eine Menge von Subinstanzen
deren gruppierung speziell f�r eine Aufgabe realisiert sein kann. Die
entstandenen Mengen sind zueinander disjunkt. Weiterhin soll f�r jede
Gruppe eine Karte (Labelbild) [map] generiert werden in der jedes
Subinstanzen-Individuum mit einem eigenen Label passend zur Beschreibung
in der Gruppe eingetragen ist. F�r alle Gruppen ist ebenfalls eine
Karte [file] zu erstellen, in der jede Gruppe einen eigenen
eindeutigen Label besitzt.

\subsubsection{Input f�r den BottomUp-Operator}
Eine Knotenliste siehe oben...\\
Einen Prefix aus dem die Output-Datei-Namen generiert werden.\\
Die Namen f�r die \textit{map}-files - \textbf{prefix\_map\_[NR].plm}\\
Der Name f�r das \textit{file} (labelfile) - \textbf{prefix.plm}\\
Der Name f�r das Beschreibungsfile - \textbf{prefix}\\

\subsubsection{Output des BottomUp-Operators}
Diese Attribute beschreiben die Ergebnisdarstellung des
BottomUp-Operators. 
\begin{description}
\item[id]: Gruppen ID
\item[map]: Karte mit Subinstanzen
\item[file]: Karte mit den Instanzen der Gruppenkonzepte
\item[geoWest, geoNorth, geoEast, geoSouth]: bounding box
\end{description}
Weiterhin werden die Lablebilder (mehrere map, file) erzeugt

\subsubsection{Beispiel}
Input
\begin{verbatim}
<node class="house" id=1 file="label1" ...\>
<node class="house" id=2 file="label1" ...\>
<node class="house" id=3 file="label1" ...\>
<node class="house" id=4 file="label1" ...\>
\end{verbatim}

Output
\begin{verbatim}
<group 
  <node class="house" id=1 file="label1" ...\>
  <node class="house" id=2 file="label1" ...\>
  id=1 map="prefix_map_1.plm" file="prefix.plm"
\>
<group
<node class="house" id=3 file="label1" ...\>
<node class="house" id=4 file="label1" ...\>
  id=2 map="prefix_map_2.plm" file="prefix.plm"
\>
\end{verbatim}

%%% Local Variables: 
%%% mode: latex
%%% TeX-master: "guide"
%%% End: 
