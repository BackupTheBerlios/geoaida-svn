\section{Regionenbeschreibung}

Eine Regionenbeschreibung entsteht als Ergebnis eines
\emph{topDown}-Operators. Sie besteht aus einer Liste von Tags des
Typs \texttt{region}. Die folgende Attribue m�ssen/k�nnen gesetzt
werden:

\begin{description}
\item[class]  bestimmt die Klasse, von welcher diese Region ist. Nur
  Regionen, die der Klasse des aufrufenden Knotens entsprechen, werden
  in das Instanzennetz �berf�hrt.
\item[id:] Unter der angegebenen \emph{id} ist die Region im Labelbild
  zu finden.
\item[file] ist der Dateiname des zu dieser Region geh�renden
  Labelbildes. Mehrere Regionen k�nnen das gleiche Labelbild
  verwenden. Die Unterscheidung findet �ber die \emph{id} statt.
\item[llx, lly, urx, ury] beschreiben die Boundingbox der Region in
  Pixeln im angegegeben Labelbild \emph{file}. Dabei gilt:
  $llx<=x<=urx$ und $ury<=y<=lly$, d.h. die Boundingbox liegt
  vollst�ndig in der Region.
\item[geoNorth, geoSouth, geoWest, geoEast] beschreibt die Boundingbox
  der Region. Die beschriebene Boundingbox in Geokoordinaten liegt
  genau auf der Regionengrenze und ist damit in jede Richtung einen
  halben Pixel gr��er als die Boundingbox in Pixeln. Geokoordinaten
  m�ssen nicht angegeben werden. Dann wird jedoch angenommen, dass die
  Geokoordinaten und die Gr��e des Gesamtbildes (\texttt{file}) denen
  der �bergeordneten Region entsprechen.
\item[name] ist der Name der Region. Er dient lediglich
  Visualisierungszwecken. Ist kein \emph{name} angegeben, wird ein
  Name automatisch aus dem Klassenname und einer fortlaufenden Nummer
  generiert. 
\end{description}
%%% Local Variables: 
%%% mode: latex
%%% TeX-master: "guide"
%%% End: 
