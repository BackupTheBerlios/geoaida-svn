\section{Bilddatenbeschreibung}

Die Bilddaten, der in \geoaida\ verwendeten Bilder werden in einem
ASCII-Format beschrieben. Die folgende Attribue m�ssen/k�nnen gesetzt
werden:

\begin{description}
\item[file] ist der Dateiname des zu dieser Bild geh�renden
  Rasterdaten. 
\item[geoType] bestimmt die Art des Geokoordinatensystems. Bisher wird
  lediglich \texttt{GK/1}, \texttt{GK/2} und \texttt{GK/3}
  (Gau�-Kr�ger) unterst�tzt. Denkbar w�ren jedoch auch Typen wie
  \texttt{UTM}.
\item[geoNorth, geoSouth, geoWest, geoEast] beschreibt die
  Geokoordinaten der Rasterdaten im mit \texttt{geoType} festgelegten
  Koordinatensystem. Die Geokoordinaten beschreiben die Bildpunkte
  \emph{(-0.5, size\_y-0.5)} und \emph{(size\_x-0.5, -0.5)} (s. auch
  Kapitel \ref{chap:geocoordinates}).
\item[type] bestimmt den Typ dieses Bildes. Derzeit sind die Bildtypen
  \texttt{LASER}, \texttt{VIS}, \texttt{SAR},\texttt{IR} und
  \texttt{VIDEO} m�glich.  Im Prinzip ist jeder beliebige Bildtyp
  m�glich, sofern in der Datenbank ebenfalls Bilder diesen Typs f�r
  die angew�hlte Region existieren.
\item[key] dient dazu, das Bild innerhalb seiner \texttt{type}-Klasse
  eindeutig zu identifizieren.
\item[size\_x, size\_y]
\item[res\_x, res\_y]  
\item[name] ist der Name des Bildes.  
\end{description}

%%% Local Variables: 
%%% mode: latex
%%% TeX-master: "guide"
%%% End: 
