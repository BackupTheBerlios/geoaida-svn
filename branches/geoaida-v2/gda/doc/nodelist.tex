\section{Knotenliste}

Eine Knotenliste wird an einen \emph{bottomUp}-Opertator �bergeben,
dessen Aufgabe darin besteht:
\begin{enumerate}
\item die Knoten zu gruppieren,
\item eine Karte der Knoten zu generieren,
\item die Knoten einer Karte zu einer Gruppenregion zusammenzufassen.
\end{enumerate}

Knoten werden mittel \verb$\<node$ eingeleitet, Gruppen mittels
\verb$\<group$.

Als Eingabe erh�lt der \texttt{bottomUp}-Operator eine List von
\emph{node}s. \texttt{node}s k�nnen beliebige Attribute haben. Die
folgenden Attribute sind in jedem Fall vorhanden.
\begin{description}
\item[class]  bestimmt die Klasse, von welcher diese Region ist.
\item[id:] Unter der angegebenen \emph{id} ist die Region im Labelbild
  zu finden.
\item[file] ist der Dateiname des zu dieser Region geh�renden
  Labelbildes. Mehrere Regionen k�nnen das gleiche Labelbild
  verwenden. Die Unterscheidung findet �ber die \emph{id} statt.
\item[llx, lly, urx, ury] beschreiben die Boundingbox der Region in
  Pixeln im angegegeben Labelbild \emph{file}. Dabei gilt:
  $llx<=x<=urx$ und $ury<=y<=lly$, d.h. die Boundingbox liegt
  vollst�ndig in der Region.
\item[geoNorth, geoSouth, geoWest, geoEast] beschreibt die Boundingbox
  der Region. Die beschriebene Boundingbox in Geokoordinaten liegt
  genau auf der Regionengrenze und ist damit in jede Richtung einen
  halben Pixel gr��er als die Boundingbox in Pixeln.
\item[file\_geoNorth, file\_geoSouth, file\_geoWest, file\_geoEast]
  beschreibt die Boundingbox des mittels \texttt{file} angegebenen
  Label-Bildes.
%\item[llx, lly, urx, ury] beschreiben die Boundingbox der Region in
%  Pixeln im angegegeben Labelbild \emph{file}.
%\item[geoNorth, geoSouth, geoWest, geoEast] beschreibt die Boundingbox
%  der Region in Geokoordinaten.
\item[name] ist der Name der Region. Er dient lediglich
  Visualisierungszwecken. Ist kein \emph{name} angegeben, wird ein
  Name automatisch aus dem Klassenname und einer fortlaufenden Nummer
  generiert. 
\item[addr] gibt die Adresse dieses Attributs im Speicher an. Dieses
  Attribut darf in keinem Fall ver�ndert werden, da es die Zuordnung
  dieses Knotens im \geoaida-Netz herstellt.
\end{description}

In den einzelnen Knoten muss der \emph{bottomUp}-Operator die
folgenden Attribute setzen:
\begin{description}
\item[id] 
\end{description}

\texttt{group}-Attribute:

\begin{description}
\item[map] 
\item[file] 
\item[id] 
\end{description}

%%% Local Variables: 
%%% mode: latex
%%% TeX-master: "guide"
%%% End: 
