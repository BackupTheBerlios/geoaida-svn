
\chapter{Dateiformate}
\label{cha:asciifileformat}

Der Informationsaustausch zwischen den einzelnen Kompononten von
\geoaida geschieht �ber ASCII-Dateien, in denen die Daten in einem
HTML-�hnlichen Format gespeichert werden. Dadurch lassen sich
hierarchische Strukturen einfach darstellen. Jeder Knoten der
Hierarchie wird durch einen Start-Tag eingeleitet und durch einen
End-Tag beendet. Zwischen Start- und End-Tag k�nnen Child-Knoten
beschrieben werden, die ihrerseits geschachtelt sein k�nnen.

\textbf{Start-Tags} werden mit \texttt{<} eingeleitet. Unmittelbar auf das
\texttt{<} folgt der Name des Tags. Dem Name des Tags kann eine Liste
von Attributen folgen, die dem zu beschreibenen Knoten bestimmte
Eigenschaften zuordnen. Das Start-Tag wird durch \texttt{>}
beendet. Enth�lt der Knoten keine Child-Knoten, so kann das Start-Tag
auch mit \texttt{/>} abgeschlossen werden. Es ist dann kein End-Tag
mehr erforderlich.

Das \textbf{End-Tag} wird mit \texttt{</} eingeleitet. Darauf folgt
der Name des End-Tags der mit einem zuvor benutzten Start-Tag
korrespondieren muss.

Die beim Start-Tag einsetzbare \textbf{Attributeliste} hat das Format
\emph{Attribute-Name}\texttt{=}\emph{Value}, wobei mehrere Attribute
durch Leerzeichen oder andere Widespaces getrennt aufeinander folgen
k�nnen. \emph{Attribute-Name} und \emph{Value} d�rfen aus beliebigen
Zeichen bestehen. Werden Zeichen verwendet, die auch zum Einleitung
oder Beenden der Tags genutzt werden, muss der Ausdruck in
Anf�hrungszeichen gesetzt werden. Dem Zeichen \verb$"$ muss ein
Backslash \texttt{$\backslash$} vorangestellt werden, gleiches gilt
f�r nat�rlich f�r das Zeichen \texttt{$\backslash$} selber.

Im folgenden Beispiel sind insgesamt drei Knoten dargestellt. Zwei vom
Typ \texttt{region} und einer vom Typ \texttt{subregion}.
\texttt{subregion} ist Child des \texttt{region}-Knoten mit dem Name
Knoten2. Der \texttt{region}-Knoten mit dem Namen Knoten1 sowie der
\texttt{subregion}-Knoten sind einfache Knoten ohne Children. Deshalb
k�nnen deren Beschreibungen mit \texttt{/>} beendet werden, wohingegen
der zweite \texttt{region}-Knoten ein Endtag (\texttt{</region>})
ben�tigt. Die beiden \texttt{region}-Knoten besitzen jeweils die
Attribute \emph{id} und \emph{name}, \texttt{subregion} besitzt
\emph{id} und \emph{label}. Das Label Attribut zeigt dabei, wie
Sonderzeichen durch Nutzung des Escape-Zeichen \texttt{$\backslash$}
in die Attributewerte integriert werden k�nnen.

\begin{verbatim}
<region id=5 name="Knoten1"/>
<region id=3 name=Knoten2>
  <subregion id=6 label="Baum \"Oskar\""/>
</region>
\end{verbatim}

\section{Regionenbeschreibung}

Eine Regionenbeschreibung entsteht als Ergebnis eines
\emph{topDown}-Operators. Sie besteht aus einer Liste von Tags des
Typs \texttt{region}. Die folgende Attribue m�ssen/k�nnen gesetzt
werden:

\begin{description}
\item[class]  bestimmt die Klasse, von welcher diese Region ist. Nur
  Regionen, die der Klasse des aufrufenden Knotens entsprechen, werden
  in das Instanzennetz �berf�hrt.
\item[id:] Unter der angegebenen \emph{id} ist die Region im Labelbild
  zu finden.
\item[file] ist der Dateiname des zu dieser Region geh�renden
  Labelbildes. Mehrere Regionen k�nnen das gleiche Labelbild
  verwenden. Die Unterscheidung findet �ber die \emph{id} statt.
\item[llx, lly, urx, ury] beschreiben die Boundingbox der Region in
  Pixeln im angegegeben Labelbild \emph{file}. Dabei gilt:
  $llx<=x<=urx$ und $ury<=y<=lly$, d.h. die Boundingbox liegt
  vollst�ndig in der Region.
\item[geoNorth, geoSouth, geoWest, geoEast] beschreibt die Boundingbox
  der Region. Die beschriebene Boundingbox in Geokoordinaten liegt
  genau auf der Regionengrenze und ist damit in jede Richtung einen
  halben Pixel gr��er als die Boundingbox in Pixeln. Geokoordinaten
  m�ssen nicht angegeben werden. Dann wird jedoch angenommen, dass die
  Geokoordinaten und die Gr��e des Gesamtbildes (\texttt{file}) denen
  der �bergeordneten Region entsprechen.
\item[name] ist der Name der Region. Er dient lediglich
  Visualisierungszwecken. Ist kein \emph{name} angegeben, wird ein
  Name automatisch aus dem Klassenname und einer fortlaufenden Nummer
  generiert. 
\end{description}
%%% Local Variables: 
%%% mode: latex
%%% TeX-master: "guide"
%%% End: 

\section{Bilddatenbeschreibung}

Die Bilddaten, der in \geoaida\ verwendeten Bilder werden in einem
ASCII-Format beschrieben. Die folgende Attribue m�ssen/k�nnen gesetzt
werden:

\begin{description}
\item[file] ist der Dateiname des zu dieser Bild geh�renden
  Rasterdaten. 
\item[geoType] bestimmt die Art des Geokoordinatensystems. Bisher wird
  lediglich \texttt{GK/1}, \texttt{GK/2} und \texttt{GK/3}
  (Gau�-Kr�ger) unterst�tzt. Denkbar w�ren jedoch auch Typen wie
  \texttt{UTM}.
\item[geoNorth, geoSouth, geoWest, geoEast] beschreibt die
  Geokoordinaten der Rasterdaten im mit \texttt{geoType} festgelegten
  Koordinatensystem. Die Geokoordinaten beschreiben die Bildpunkte
  \emph{(-0.5, size\_y-0.5)} und \emph{(size\_x-0.5, -0.5)} (s. auch
  Kapitel \ref{chap:geocoordinates}).
\item[type] bestimmt den Typ dieses Bildes. Derzeit sind die Bildtypen
  \texttt{LASER}, \texttt{VIS}, \texttt{SAR},\texttt{IR} und
  \texttt{VIDEO} m�glich.  Im Prinzip ist jeder beliebige Bildtyp
  m�glich, sofern in der Datenbank ebenfalls Bilder diesen Typs f�r
  die angew�hlte Region existieren.
\item[key] dient dazu, das Bild innerhalb seiner \texttt{type}-Klasse
  eindeutig zu identifizieren.
\item[size\_x, size\_y]
\item[res\_x, res\_y]  
\item[name] ist der Name des Bildes.  
\end{description}

%%% Local Variables: 
%%% mode: latex
%%% TeX-master: "guide"
%%% End: 

\section{Knotenliste}

Eine Knotenliste wird an einen \emph{bottomUp}-Opertator �bergeben,
dessen Aufgabe darin besteht:
\begin{enumerate}
\item die Knoten zu gruppieren,
\item eine Karte der Knoten zu generieren,
\item die Knoten einer Karte zu einer Gruppenregion zusammenzufassen.
\end{enumerate}

Knoten werden mittel \verb$\<node$ eingeleitet, Gruppen mittels
\verb$\<group$.

Als Eingabe erh�lt der \texttt{bottomUp}-Operator eine List von
\emph{node}s. \texttt{node}s k�nnen beliebige Attribute haben. Die
folgenden Attribute sind in jedem Fall vorhanden.
\begin{description}
\item[class]  bestimmt die Klasse, von welcher diese Region ist.
\item[id:] Unter der angegebenen \emph{id} ist die Region im Labelbild
  zu finden.
\item[file] ist der Dateiname des zu dieser Region geh�renden
  Labelbildes. Mehrere Regionen k�nnen das gleiche Labelbild
  verwenden. Die Unterscheidung findet �ber die \emph{id} statt.
\item[llx, lly, urx, ury] beschreiben die Boundingbox der Region in
  Pixeln im angegegeben Labelbild \emph{file}. Dabei gilt:
  $llx<=x<=urx$ und $ury<=y<=lly$, d.h. die Boundingbox liegt
  vollst�ndig in der Region.
\item[geoNorth, geoSouth, geoWest, geoEast] beschreibt die Boundingbox
  der Region. Die beschriebene Boundingbox in Geokoordinaten liegt
  genau auf der Regionengrenze und ist damit in jede Richtung einen
  halben Pixel gr��er als die Boundingbox in Pixeln.
\item[file\_geoNorth, file\_geoSouth, file\_geoWest, file\_geoEast]
  beschreibt die Boundingbox des mittels \texttt{file} angegebenen
  Label-Bildes.
%\item[llx, lly, urx, ury] beschreiben die Boundingbox der Region in
%  Pixeln im angegegeben Labelbild \emph{file}.
%\item[geoNorth, geoSouth, geoWest, geoEast] beschreibt die Boundingbox
%  der Region in Geokoordinaten.
\item[name] ist der Name der Region. Er dient lediglich
  Visualisierungszwecken. Ist kein \emph{name} angegeben, wird ein
  Name automatisch aus dem Klassenname und einer fortlaufenden Nummer
  generiert. 
\item[addr] gibt die Adresse dieses Attributs im Speicher an. Dieses
  Attribut darf in keinem Fall ver�ndert werden, da es die Zuordnung
  dieses Knotens im \geoaida-Netz herstellt.
\end{description}

In den einzelnen Knoten muss der \emph{bottomUp}-Operator die
folgenden Attribute setzen:
\begin{description}
\item[id] 
\end{description}

\texttt{group}-Attribute:

\begin{description}
\item[map] 
\item[file] 
\item[id] 
\end{description}

%%% Local Variables: 
%%% mode: latex
%%% TeX-master: "guide"
%%% End: 

\section{Mapfileformat}

\subsubsection{Map}

\begin{tabular}{|l|l|l|}
\hline
\emph{Typ} & \emph{Bezeichnung} & \emph{Beschreibung} \\
\hline
DD & num\_regions & Anzahl der Regionen \\
\hline
Region & region\_1 & 1. Region \\
\vdots & \vdots & \\
Region & region\_n & letzte Region \\
\hline
VectorList& vectorlist\_1 & 1. Vektorliste \\
\vdots & \vdots & \\
VectorList& vectorlist\_n & letzte Vektorliste \\
\hline
\end{tabular}

\subsubsection{Region}
\begin{tabular}{|l|l|l|}
\hline
\emph{Typ} & \emph{Bezeichnung} & \emph{Beschreibung} \\
\hline
DD & id & ID der Region \\
\hline
DD & file\_offset & Position der Vektorliste in der Datei relativ zum
Start der Vektorlisten\\
\hline
\end{tabular}

\subsubsection{VectorList}
\begin{tabular}{|l|l|l|}
\hline
\emph{Typ} & \emph{Bezeichnung} & \emph{Beschreibung} \\
\hline
DD & num\_vectors & Anzahl der Vektoren \\
\hline
Vector& vector\_1 & 1. Vektor \\
\vdots & \vdots & \\
Vector& vector\_n & letzter Vektor \\
\hline
\end{tabular}

\subsubsection{Vector}
\begin{tabular}{|l|l|l|}
\hline
\emph{Typ} & \emph{Bezeichnung} & \emph{Beschreibung} \\
\hline
DD & start\_x & Startpunkt X \\
DD & start\_y & Startpunkt Y \\
DD & num\_points & Anzahl der Richtungsanweisungen \\
\hline
Direction & dir\_1 & 1. Richtung \\
Direction & dir\_n & letzte Richtung \\
\hline
\end{tabular}

\subsubsection{Direction}
\emph{Direction} ist ein 2 bit-Wert mit den folgenden Bedeutungen:

\begin{tabular}{ll}
0 & East \\
1 & North \\
2 & West \\
3 & South \\
\end{tabular}

Jeweils 4 Richtungswerte werden in einem Byte zusammengefasst:

\begin{tabular}{lc@{}c@{}c@{}c}
Bit&7\ 6&5\ 4&3\ 2&1\ 0\\
&$\underbrace{x\ x}$&$\underbrace{x\ x}$&$\underbrace{x\ x}$&$\underbrace{x\ x}$\\
Dir-Index&3 &2 &1 &0 \\
\end{tabular}
%%% Local Variables: 
%%% mode: latex
%%% TeX-master: "guide"
%%% End: 

%%% Local Variables: 
%%% mode: latex
%%% TeX-master: "guide"
%%% End: 
