\section{Mapfileformat}

\subsubsection{Map}

\begin{tabular}{|l|l|l|}
\hline
\emph{Typ} & \emph{Bezeichnung} & \emph{Beschreibung} \\
\hline
DD & num\_regions & Anzahl der Regionen \\
\hline
Region & region\_1 & 1. Region \\
\vdots & \vdots & \\
Region & region\_n & letzte Region \\
\hline
VectorList& vectorlist\_1 & 1. Vektorliste \\
\vdots & \vdots & \\
VectorList& vectorlist\_n & letzte Vektorliste \\
\hline
\end{tabular}

\subsubsection{Region}
\begin{tabular}{|l|l|l|}
\hline
\emph{Typ} & \emph{Bezeichnung} & \emph{Beschreibung} \\
\hline
DD & id & ID der Region \\
\hline
DD & file\_offset & Position der Vektorliste in der Datei relativ zum
Start der Vektorlisten\\
\hline
\end{tabular}

\subsubsection{VectorList}
\begin{tabular}{|l|l|l|}
\hline
\emph{Typ} & \emph{Bezeichnung} & \emph{Beschreibung} \\
\hline
DD & num\_vectors & Anzahl der Vektoren \\
\hline
Vector& vector\_1 & 1. Vektor \\
\vdots & \vdots & \\
Vector& vector\_n & letzter Vektor \\
\hline
\end{tabular}

\subsubsection{Vector}
\begin{tabular}{|l|l|l|}
\hline
\emph{Typ} & \emph{Bezeichnung} & \emph{Beschreibung} \\
\hline
DD & start\_x & Startpunkt X \\
DD & start\_y & Startpunkt Y \\
DD & num\_points & Anzahl der Richtungsanweisungen \\
\hline
Direction & dir\_1 & 1. Richtung \\
Direction & dir\_n & letzte Richtung \\
\hline
\end{tabular}

\subsubsection{Direction}
\emph{Direction} ist ein 2 bit-Wert mit den folgenden Bedeutungen:

\begin{tabular}{ll}
0 & East \\
1 & North \\
2 & West \\
3 & South \\
\end{tabular}

Jeweils 4 Richtungswerte werden in einem Byte zusammengefasst:

\begin{tabular}{lc@{}c@{}c@{}c}
Bit&7\ 6&5\ 4&3\ 2&1\ 0\\
&$\underbrace{x\ x}$&$\underbrace{x\ x}$&$\underbrace{x\ x}$&$\underbrace{x\ x}$\\
Dir-Index&3 &2 &1 &0 \\
\end{tabular}
%%% Local Variables: 
%%% mode: latex
%%% TeX-master: "guide"
%%% End: 
